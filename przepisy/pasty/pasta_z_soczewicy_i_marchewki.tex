	\section{Pasta z soczewicy i marchewki}
	\begin{table}[h]
	% \centering
	\begin{tabular}{l l l}
	\toprule
	\textbf{składniki} & \\
	\midrule
	soczewica & 400 gramów \\
	marchewka & 4 sztuki \\
	cebula 	  & 2 sztuka \\
	olej kososowy & 2 łyżeczki \\
	\midrule
	\textbf{przyprawy}&\\
	\midrule
	słodka papryka & 2 łyżeczki\\
	curry & 2 łyżeczki\\
	sól & do smaku\\
	\bottomrule
	\end{tabular}
	% \caption{}

	\end{table}
	\begin{exercise}[ok. 30 minut] wymagania: 2 palniki, garnek, sito, patelnia, miska
	\begin{description}
	\item[soczewicę] ugotować aż będzie odpowiednio\footnote{nie pozwólmy tu jednak wkradać si żadnym relatywizmom, ma być odpowiednio i już} papkowata, po czym odsączyć i wrzucić z powrotem do garnka
	\item[marchewkę] zetrzeć i dusić na oleju aż zmięknie, dodać czosnek (posiekany lub przeciśnięty przez praskę\footnote{której mycie trwa zasadniczo dłużej niż siekanie czosnku, ale to moja opinia}), paprykę, curry i sól po czym śmiałym gestem wrzucić do soczewicy
	\item[cebulę] drobno pokroiwszy dusić na średnim ogniu aby miękkości słusznej jej przydać po czym wszystko wymieszać razem i przez kilka minut podgrzewać by się wszystko ładnie połączyło, można -- a nawet trzeba -- zmodyfikować smak potrawy przyprawami wedle uznania
	\item[pastę] która jest już niemalże gotowa można zmiksować dla nadania jednolitej konsystencji -- albo i nie, tak czy siak będzie pyszna
	\end{description}
	\end{exercise}