\documentclass[./KNIGA.tex]{subfiles}
\begin{document}
\subparagraph{Wok} jaki jest, każdy widzi. Ten rodzaj chińskiej patelni, za sprawą filmów, a także przydrożnych budek\footnote{prawo sformułowane przez jednego z przyjaciół mówi, że im gorzej chńczyk wygląda, tym lepsze jedzenie, cos w tym jest\ldots} z tanią strawą, na stałe wrósł już w naszą kuchnię. I dobrze :)
\subparagraph{Smażenie na woku} polega na dostarczeniu maksymalnej ilości energii cieplnej w dość krótkim czasie, potrzeba zatem naprawdę sporego palnika który dał by radę, minimium 9kW mocy... Tyle maję mniej więcej palniki używane w chińskich knajpkach. Te w standardowych kuchenkach gazowych mają ~1,5kW, zatem to trochę za mało... rozwiązaniem jest smazyć niewielkie ilości na dobrze rozgrzanej patelni albo nabyć porządny palnik/taboret gazowy. Cena koło 300PLN za palnik firmy Foker. Testowałem. Jest niezły :)
\subparagraph{Sezonowanie woka} - przed użyciem należy patelnie pokryć nieprzywierającą ochronną warstwą węgla. Robi się to w bardzo prosty sposób: smarujemy woka olejem, stawiamy na największy ogien albo wręcz do piekarnika... Pamiętajmy by uprzedzić sąsiadów i Straż Pożarną, jako że dymu jest co niemiara...
\subparagraph{Mycie woka} - również bardzo proste: stawiamy wok na ogień, wlewamy trochę wody i szorujemy druciakiem przymocowanym do jakiegoś kijka (osobiście uzywam tak zwanej mątewki za ~2PLN). Potem płuczemy zimna wodą i dokładnie suszymy nad gazem. Po wszystkim warto jeszcze posmarowac olejem co by patelnia nie rdzewiała.
\end{document}