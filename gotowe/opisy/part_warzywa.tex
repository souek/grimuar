\documentclass[./KNIGA.tex]{subfiles}
\begin{document}
\subparagraph{Warzywo:}
w języku staropolskim oznaczało po prostu potrawę gotowaną, warzyć oznaczało gotować\footnote{i znów nasza kochana ortografia}, do naszych czasów przetrwalo słowo ,,warnik'' oznaczające przyrząd do robienia wrzątku.

\end{document}