\documentclass[./KNIGA.tex]{subfiles}
\begin{document}
\subparagraph{$C_6H_{12}O_6$:}
\emph{fruktoza} jest to jeden z tak zwanych cukrów prostych\footnote{o ile pamiętam jej bliźniaczka \emph{glukoza} ma ten sam wzór chemiczny tylko cząsteczka jej skręca się w drugą stronę}, naturalnie występujących w owocach, miodzie i tak dalej. W tym przypadku w tyej podsekcji zajmę się odmianami owsianki ,,na słodko''.
\end{document}