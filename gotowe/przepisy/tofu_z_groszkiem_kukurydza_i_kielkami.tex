\documentclass[./KNIGA.tex]{subfiles}
\begin{document}
\recipe{Tofu z groszkiem, kukurydzą, i kiełkami}
\ingred{\begin{tabular}{lll}
	\textbf{składniki na 2-3 osoby} & \textbf{} & \textbf{cena w PLN}\\
	tofu & 1k kostka & 4\\
	cebula cukrowa & 1 sztuka & 0,5\\
	groszek zielony & 1/2 puszki & 0,75\\
	kukurydza & 1/2 puszki & 0,75\\
	pomidor & 2 sztuki & 1\\
	zioła prowansalskie, rozmaryn\\ zielona pasta curry\\ ew.sól & do smaku & 0,5\\
	& łączna kwota & 7,5 PLN
\end{tabular}}
\begin{description}
\item [tofu] odsącz z “serwatki” i pokrój w kostkę
\item [cebulę] obierz i pokrój w półtalarki
\item [kiełki] opłucz i pokrój na odcinki ok 3 cm
\item [groszek i kukurydzę] odsącz z zalewy
\item [pomidory] umyj, przekrój na pół i wykrój środek, po czym pokrój w kostkę
\item [tofu] wrzuć do rozgrzanego woka wlej trochę oliwy, (pamiętaj żeby cały czas mieszać bo lubi się przykleić), a gdy się podrumieni, cebulę
\item gdy cebula zacznie robić się miękka dodaj pół łyżeczki pasty curry, groszek, kukurydzę, kiełki i zioła
\item po kilku minutach zestaw z ognia i wymieszaj z pomidorami
\item smacznego 
\end{description}
\end{document}